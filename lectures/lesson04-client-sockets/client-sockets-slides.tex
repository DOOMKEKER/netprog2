
\startcomponent client-sockets-slides
\environment slides-env

\setupTitle
  [ title={ПО сетевых устройств},
   author={Трещановский Павел Александрович, к.т.н.},
     date={\date},
  ]
\placeTitle

\SlideTitle {Компьютерные сети}
\IncludePicture[horizontal][diagrams/network.pdf]

\SlideTitle {Локальная сеть Ethernet}
\startitemize
\item Локальная сеть Ethernet состоит из конечных узлов, соединенных
коммутаторами.
\item Сеть передает Ethernet-кадры.
\item Каждый конечный узел имеет уникальный MAC-адрес. Пример:
68:eb:c5:00:01:82.
\item Простота - не требуются протоколы маршрутизации для доставки кадров.
\item Адресат кадра обнаруживается с помощью широковещательных рассылок.
\item Множество сред передачи: медь, оптика, радио.
\stopitemize

\SlideTitle {Заголовок Ethernet-кадра}
\IncludePicture[horizontal][diagrams/ethernet-header.pdf]

\SlideTitle {Глобальная сеть IP}
\startitemize
\item Глобальная сеть IP состоит из локальных сетей, соединенных маршрутизаторами.
\item Сеть передает IP-дейтаграммы.
\item Каждый конечный узел имеет уникальный IP-адрес. Пример: 64.233.162.113.
\item Глобальные широковещательные рассылки запрещены.
\item Требуются протоколы маршрутизации для нахождения адресата.
\item Для передачи по локальной сети используется Ethernet или аналогичная
технология.
\stopitemize

Перевод из сетевого порядка в процессорный (network to host):
\starttyping
processor_int = htonl(network_int);
processor_short = htons(network_short);
\stoptyping

\SlideTitle {Заголовок IP-дейтаграммы}
\IncludePicture[horizontal][diagrams/ip-header.png]

\SlideTitle {Инкапсуляция}
\IncludePicture[horizontal][diagrams/encapsulation.png]

\SlideTitle {Проблемы сетей Ethernet/IP}
\startitemize
\item Возможна потеря пакетов из-за помех в среде передачи или переполнения
буферов коммутаторов/маршрутизаторов.
\item Возможно изменение порядка пакетов при передаче из-за переходных
процессов при изменении таблиц маршрутизации.
\item Отсутствие гарантий по максимально возможной задержке доставки.
\stopitemize

\SlideTitle {Повторная отправка пакета}
\IncludePicture[horizontal][diagrams/sequence-numbering.pdf]

\SlideTitle {Протокол TCP}
\startitemize
\item Протокол соединяет удаленные приложения.
\item Приложение идентифицируется номером порта (числа от 1 до 65535).
\item Сообщения протокола называются {\em сегментами}.
\item Каждый передаваемый байт нумеруется. Принимающая сторона должна
подтвердить прием отправкой ACK. При отсутствии подтверждения выполняется
повторная отправка.
\item Передача данных требует установки соединения для синхронизации ожидаемых
номеров байтов.
\stopitemize

\SlideTitle {Установка соединения}
\IncludePicture[horizontal][diagrams/tcp-connection.pdf]

\SlideTitle {Особенности TCP}
\startitemize
\item Протокол передает поток данных, разбитый на {\em сегменты}. Сохранение
границ сегментов не гарантируется.
\item Типичные приложения: передача файлов (FTP), удаленный терминал (SSH),
передача почтовых сообщений (SMTP).
\item Повторная отправка может значительно увеличивать время доставки.
\item Повторная отправка съедает пропускную способность сети. При наличии
множества потоков требуется ограничивать скорость передачи сообщений.
\stopitemize

\SlideTitle {Границы сегмента}
\IncludePicture[horizontal][diagrams/segment-boundaries.pdf]

\SlideTitle {4-уровневая модель сетевого взаимодействия}
\IncludePicture[horizontal][diagrams/4-layer-model.pdf]

\SlideTitle {4-уровневая модель, реализация}
\IncludePicture[horizontal][diagrams/4-layer-model-implementation.pdf]

\SlideTitle {Сокеты}
\starttyping
int sock_fd;

/* int socket(int domain, int type, int protocol); */
sock_fd = socket(AF_INET, SOCK_STREAM, 0);
if (sock_fd < 0)
	goto on_error;
\stoptyping

\startitemize
\item {\tt domain} - семейство адресов. {\tt AF_INET} - IPv4, {\tt AF_INET6} -
IPv6 и др.
\item {\tt type} - тип сокета. Обычно либо потоковый сокет ({\tt SOCK_STREAM})
или дейтаграммный сокет ({\tt SOCK_DGRAM}).
\item {\tt protocol} - протокол передачи данных. 0 - протокол по умолчанию. Для
потоковых IP-сокетов протокол по умолчанию - TCP.
\stopitemize

\SlideTitle {Установка соединения, API}
\starttyping
int sock_fd;
struct sockaddr_in addr;
int ret;

addr.sin_family = AF_INET;
/* Для номера порта и абреса используется сетевой порядок байтов! */
add.sin_port = htons(21);
inet_aton("127.0.0.1", &addr.sin_addr);

sock_fd = socket(AF_INET, SOCK_STREAM, 0);

/* int connect(int sockfd, struct sockaddr *serv_addr, socklen_t addrlen); */
ret = connect(sock_fd, (struct sockaddr *)&addr, sizeof(addr));
if (ret < 0)
	goto on_error;
\stoptyping

\SlideTitle {Передача данных, API}

\starttyping
int sock_fd;
char out_buf[] = {2, 0, 2, 0};
char in_buf[100];
int sz;

/* int send(int s, const void *buf, size_t len, int flags); */
sz = send(sock_fd, out_buf, sizeof(out_buf), 0);
if (sz < 0)
	goto on_error;

sz = recv(sock_fd, in_buf, sizeof(in_buf), 0);

printf("Received %d bytes from socket\n", sz);

\stoptyping

\SlideTitle {Передача данных через поток ввода-вывода}

\starttyping
int sock_fd;
FILE *sock_stream;
char response[100];

sock_stream = fdopen(sock_fd, "r+");
if (!sock_stream)
	goto on_error;

ret = fprintf(sock_stream, "request %s %d\n", "abc", 555);
if (ret < 0)
	goto on_error;

fflush(sock_stream);

fgets(response, sizeof(response), sock_stream);
\stoptyping

\SlideTitle {Протокол UDP}
Создание сокета:
\starttyping
sock_fd = socket(AF_INET, SOCK_DGRAM, 0);
\stoptyping

Особенности.
\startitemize
\item Отсутствуют гарантии доставки.
\item Сохраняются границы дейтаграмм при передаче через сеть.
\item Не вносится дополнительная задержка.
\item Адресация аналогична TCP - IP-адрес + порт.
\item Общий API с TCP, но смысл функций ({\tt connect}, {\tt send}, {\tt recv}
и др.) отличается.
\stopitemize

\SlideTitle {Unix-сокеты}
Создание сокета:
\starttyping
sock_fd1 = socket(AF_UNIX, SOCK_STREAM, 0);
sock_fd2 = socket(AF_UNIX, SOCK_DGRAM, 0);
\stoptyping

\startitemize
\item Передача только внутри между приложениями одной ОС.
\item Адрес приложения - строка. IP-адреса и порты не используются.
\item Нет передачи через сеть - нет потерь, повторных отправок и т.д.
\item Поддерживается как потоковый, так и дейтаграммный режим.
\stopitemize

\SlideTitle {Использования Unix-сокетов в Linux}
\IncludePicture[horizontal][diagrams/unix-sockets-in-linux.pdf]

\SlideTitle {Клиентские сокеты в сетевых устройствах}
UDP-сокеты:
\startitemize
\item DNS - протокол получения IP-адреса из доменного имени (например,
www.google.com или www.angtel.ru).
\item DHCP - протокол автоматической настройки IP-адреса.
\item NTP - протокол автоматической настройки системного времени.
\item SNMP - отправка сообщений об изменении состояния системы.
\stopitemize

TCP-сокеты:
\startitemize
\item OSPF, BGP - протоколы маршрутизации.
\item FTP - передача файлов.
\stopitemize

\SlideTitle {Протокол FTP в сетевых устройствах}
\IncludePicture[horizontal][diagrams/ftp-usage.pdf]

\SlideTitle {Протокол FTP, архитектура}
\IncludePicture[horizontal][diagrams/ftp-architecture.pdf]

\SlideTitle {Команды и ответы}
Синтаксис команды (аргумент нужен не для всех команд):
\starttyping
<имя команды> <аргумент>\r\n
\stoptyping
Команда скачивания файла:
\starttyping
RETR file1\r\n
\stoptyping
Синтаксис ответа:
\starttyping
<код> <сообщение>\r\n
\stoptyping
Пример ответа при успешном выполнении:
\starttyping
200 Command okay\r\n
\stoptyping
Коды 2xx - успешное выполнение, 4xx и 5xx - ошибка, 3xx - необходимо
дополнительное действие.

\SlideTitle {Аутентификация в FTP}
Отправка имени пользователя:
\starttyping
USER root\r\n
\stoptyping
Если имя правильное, получаем ответ:
\starttyping
331 User name okay, need password\r\n
\stoptyping
Необходим пароль. Передача пароля:
\starttyping
PASS 123\r\n
\stoptyping
Если пароль правильный, получаем ответ:
\starttyping
230 User logged in, proceed\r\n
\stoptyping

\SlideTitle {Передача данных}
Для передачи данных необходимо дополнительное TCP-соединение. Запрашиваем
открытие сокета на стороне сервера:
\starttyping
PASV\r\n
\stoptyping
Ответ содержит адрес и номер порта серверного сокета:
\starttyping
227 PASV ok (192,168,0,8,159,107)
\stoptyping
Каждое число в скобках - 1 байт. Первые 4 - IP-адрес, последние 2 - номер
порта.

Устанавливаем соединение для передачи данных. Далее даем команду на скачивание:
\starttyping
RETR file1\r\n
\stoptyping
Ответ:
\starttyping
226 Operation successful\r\n
\stoptyping
После этого данные можно принимать через ранее открытое соединение.

\SlideTitle {FTP-команда, пример}
\starttyping
FILE *sock_stream;
char *user_name = "root;
char response[100];
int ftp_code;
char *first_word;

ret = fprintf(sock_stream, "USER %sr\n", user_name);
if (ret < 0)
	goto on_error;
fflush(sock_stream);

fgets(response, sizeof(response), sock_stream);
/* Формат %ms означает чтение строки в динамически выделенный буфер. */
ret = sscanf(response, "%d %ms", &code, &first_word);
if (ret != 2)
	goto on_error;
\stoptyping

\SlideTitle {Прочие команды}
\startitemize
\item PWD - вывод текущего каталога FTP-сервера,
\item CWD - смена текущего каталога,
\item LIST - вывод содержимого текущего каталога,
\item MKD - создание нового каталога,
\item QUIT - завершение сессии.
\stopitemize

\SlideTitle {Стандартный клиент}
\starttyping
$ ftp 192.168.0.8
Connected to 192.168.0.8 (192.168.0.8).
220 Operation successful
Name (192.168.0.8:pavel): root
530 Login with USER+PASS
331 Specify password
Password:
230 Operation successful
ftp> passive
Passive mode on.
ftp> get init.sh
local: init.sh remote: init.sh
227 PASV ok (192,168,0,8,141,187)
150 Opening BINARY connection for init.sh (654 bytes)
226 Operation successful
654 bytes received in 0,000176 secs (3,6e+03 Kbytes/sec)
ftp>
\stoptyping

\SlideTitle {C-строка vs строка текста}
\startitemize
\item C-строка - последовательность символом, ограниченная нулем: {\tt 's',
't', 'r', 'i', 'n', 'g', '\textbackslash 0'}. Используется только внутри приложения.
\item Строка текста - последовательность символов, ограниченная символом
перехода на новую строку: {\tt 's', 't', 'r', 'i', 'n', 'g', '\textbackslash
n'}. Используется в файлах и пакетах.
\item В данных, которые находятся в файле или пакете, никогда нет нулевого
окончания. Его должна добавлять программа.
\item Некоторые стандартные функции ({\tt read()}, {\tt fread()}) используют
массив байтов. Они не используют и не добавляют нулевое окончание.
\item Некоторые стандартные функции автоматически добавляют нулевое окончание.
Например, {\tt fgets()} читает строку текста в буфер и записывает в конец
нулевое окончание.
\stopitemize

\SlideTitle {Порядок байтов}
Как представить в памяти многобайтовое число 2864434397 (0xaabbccdd)?

\\

Порядок little-endian:

\setupTABLE[row][first][align=center]
\bTABLE
\bTABLEbody
\bTR
\bTD Адрес \eTD
\bTD 0 \eTD
\bTD 1 \eTD
\bTD 2 \eTD
\bTD 3 \eTD
\eTR
\bTR
\bTD Значение \eTD
\bTD 0xdd \eTD
\bTD 0xcc \eTD
\bTD 0xbb \eTD
\bTD 0xaa \eTD
\eTR
\eTABLEbody
\eTABLE

\\

Порядок big-endian:

\setupTABLE[row][first][align=center]
\bTABLE
\bTABLEbody
\bTR
\bTD Адрес \eTD
\bTD 0 \eTD
\bTD 1 \eTD
\bTD 2 \eTD
\bTD 3 \eTD
\eTR
\bTR
\bTD Значение \eTD
\bTD 0xaa \eTD
\bTD 0xbb \eTD
\bTD 0xcc \eTD
\bTD 0xdd \eTD
\eTR
\eTABLEbody
\eTABLE

\startitemize
\item Каждый процессор ожидает определённый порядок байтов в памяти для
корректного выполнения математических операций.
\item x86 использует порядок little-endian.
\item В сетевых протоколах используется порядок big-endian (сетевой порядок
байтов).
\stopitemize

\SlideTitle {Порядок байтов, API}
Перевод из процессорного порядка в сетевой (host to network):
\starttyping
int processor_int, network_int;
short processor_short, network_short;

network_int = htonl(processor_int);
network_short = htons(processor_short);
\stoptyping

\stopcomponent
