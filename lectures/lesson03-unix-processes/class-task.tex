
\startcomponent class-task
\environment practice-env

\subsection{Упражнения к теме \quote{Процессы ОС Linux}}

\startitemize[n]
\item Скопируйте каталог class-task-templates в общуу папку виртуальной машины.
\item Написать программу {\tt fork-process}, которая создает дочерний процесс и
дожидается его завершения. Дочерний процесс выводит свой PID и завершается с
кодом 5. Родительский процесс выводит свой PID, а также причину завершения
дочернего процесса, код завершения или сигнала, приведшего к завершению.
\item Запустите {\tt fork-process}:
\starttyping
# ./fork-process
parent PID: 9108
child PID: 9109
Child exited with status 5
\stoptyping
\item Увеличьте время жизни дочернего процесса с помощью функции {\tt sleep()}.
Запустите {\tt fork-process} и завершите его с помощью сигнала {\tt SIGTERM},
указав соответствующий PID процесса:
\starttyping
# ./fork-process &
# kill <PID>
Child was terminated by signal 15
\stoptyping
\item Напишите программу {\tt catch-signal}, которая перехватывает сигналы {\tt
SIGTERM}, {\tt SIGKILL}, {\tt SIGINT}, {\tt SIGCHLD}. В функции {\tt main()}
программа реализует бесконечный цикл чтения из терминала, который прерывается
доставкой сигнала. После каждого прерывания программа выводит имена сигналов,
прервавших программу.
\item Запустите программу {\tt catch-signal} и отправьте ей сигнал {\tt SIGINT}
с помощью комбинации Ctrl+C:
\starttyping
# ./catch-signal
^C
Interrupted: sigterm 0, sigkill 0, sigint 1, sigchld 0
\stoptyping
\item Завершите catch-signal из другого терминала:
\starttyping
# pkill -9 catch-signal
\stoptyping
\item Запустите программу {\tt catch-signal} и отправьте ей сигналы {\tt
SIGTERM} и {\tt SIGCHLD}:
\starttyping
# ./catch-signal &
# pkill -SIGTERM catch-signal
Interrupted: sigterm 1, sigkill 0, sigint 0, sigchld 0
\stoptyping
\item Удалите каталог class-task-templates из общей папки.
\stopitemize

\stopcomponent
