
\startcomponent unix-processes-slides
\environment slides-env

\setupTitle
  [ title={ПО сетевых устройств},
   author={Трещановский Павел Александрович, к.т.н.},
     date={\date},
  ]
\placeTitle

\SlideTitle {Состояния процесса}
Основные состояния:
\startitemize
\item {\tt TASK_RUNNING} (буква {\tt R} в выводе {\tt ps}) - процесс готов к
исполнению на процессоре,
\item {\tt TASK_INTERRUPTIBLE} ({\tt S}) - прерываемое ожидание события,
\item {\tt TASK_UNINTERRUPTIBLE} ({\tt D}) - непрерываемое ожидание события.
\stopitemize

Независимо от состояния процесс может в текущий момент исполняться в
\startitemize
\item пространстве ядра (kernel space) или
\item пространстве пользователя (user space).
\stopitemize

\SlideTitle {Переход между состояниями}
\IncludePicture[horizontal][diagrams/process-states.pdf]

\SlideTitle {Планировщик процессов}
\IncludePicture[horizontal][diagrams/scheduler.pdf]

\SlideTitle {Вытесняющая многозадачность}
Когда процесс перестает исполняться:
\startitemize
\item процесс запрашивает у ядра задержку (например, {\tt sleep()}),
\item процесс ожидает завершения операции ввода-вывода (например, {\tt
read()}),
\item у процесса заканчивается доля времени, выделенная планировщиком,
\item в системе появляется более приоритетный процесс.
\stopitemize

Блокирующий (системный) вызов - вызов, который переводит процесс в состояние
ожидания до завершения операции ввода-вывода. Примеры: {\tt read()}, {\tt
write()}.

\SlideTitle {Планирование отложенных действий}
\IncludePicture[horizontal][diagrams/delayed-actions.pdf]

\SlideTitle {Программное окружение процесса}
\IncludePicture[horizontal][diagrams/process-environment.pdf]

\SlideTitle {Создание процесса}
\IncludePicture[horizontal][diagrams/process-forking.pdf]

\SlideTitle {Copy on write}
\IncludePicture[horizontal][diagrams/cow.pdf]

\SlideTitle {Создание процесса, API}
\starttyping
	pid_t pid;

	pid = fork();
	if (pid < 0)
		goto on_error;

	/* В этой точке исполняется 2 процесса. */
	
	if (pid == 0) {
		/* Дочерний процесс. */
	} else {
		/* Родительский процесс. */
	}
\stoptyping

\SlideTitle {Запуск приложения}
\IncludePicture[horizontal][diagrams/execution.pdf]

\SlideTitle {Запуск приложения, API}
\starttyping
	char *argv[] = {"/bin/cat", "/home/root/file1", NULL};
	int ret;

	/* int execvp (const char *file, char *const argv[]); */
	ret = execvp(argv[0], argv);

	/* В случае успеха программа не возвращается из execvp. */

	goto on_error;

\stoptyping

\SlideTitle {Преимущество раздельного подхода}

\starttyping
	pid_t pid;

	pid = fork();

	if (pid == 0) {
		/* Дочерний процесс. */

		char *argv[] = {"/bin/cat", "/home/root/file1", NULL};

		/* .....
		 * Формирование программного окружения нового процесса:
		 * перенаправление в файл,
		 * создание конвейера (pipe),
		 * открытие устройства ввода-вывода и др.
		 * /

		execvp(argv[0], argv);
	}
\stoptyping

\SlideTitle {Конвейер в shell}
\IncludePicture[horizontal][diagrams/environment-modification.pdf]

\SlideTitle {Завершение процесса}
Возможные причины завершения процесса.
\startitemize
\item Вызов функции {\tt exit()} самим процессом, в том числе в результате
выхода из функции {\tt main()}.
\item Получение {\em сигнала} от ядра или другого процесса.
\stopitemize

Сигнал - специальный механизм межпроцессного взаимодействия. Обычно
используется для управления жизненным циклом процесса.

Примеры сигналов:
\startitemize
\item нажатие {\tt Ctrl+C} в терминале отправляет сигнал SIGINT активному
процессу,
\item недопустимое обращение к памяти приводит к отправке сигнала SIGSEGV,
\item исполнение недопустимой машинной инструкции приводит к отправке сигнала
SIGILL.
\stopitemize

\SlideTitle {Ожидание завершения, процессы-зомби}
\IncludePicture[horizontal][diagrams/waitpid.pdf]

\SlideTitle {Ожидание завершения, API}
\starttyping
	pid_t pid;
	int child_status;

	/* pid_t waitpid(pid_t pid, int *status, int options);
	 * Если pid равен -1, функция будет следить за всеми дочерними процессами.
	 * Если опции не заданы, функция будет ждать завершения
	 * одного из дочерних процессов.
	 * Если задана опция WNOHANG, функция завершится сразу.
	 */

	pid = waitpid(-1, &child_status, WNOHANG);
	if (pid == 0) {
		/* Все дочерние процессы продолжают работать. */
		return;
	}
	if (WIFEXITED(child_status))
		printf("Процесс %d завершился со статусом %d\n",
		       pid, WEXITSTATUS(child_status));
	else if (WIFSIGNALED(child_status))
		printf("Процесс %d был прерван сигналом %d\n",
		       pid, WTERMSIG(child_status));
\stoptyping

\SlideTitle {Сигналы, определение, типы}

Сигнал - механизм межпроцессного взаимодействия, который прерывает нормальное
исполнение процесса для выполнения {\em обработчика сигнала}.

Назначение некоторых важных сигналов:
\startitemize
\item SIGTERM - завершение процесса по запросу другого процесса с
освобождением всех выделенных ресурсов,
\item SIGKILL - немедленное завершение процесса,
\item SIGINT - завершение процесса со стороны управляющего терминала,
\item SIGSEGV, SIGILL, SIGABRT - завершение процесса в результате некорректных
действий самого процесса,
\item SIGCHLD - оповещение о событии в дочернем процессе.
\stopitemize

\SlideTitle {Жизненный цикл сигналов}
\IncludePicture[horizontal][diagrams/signal-lifecycle.pdf]

\SlideTitle {Доставка сигналов}
\IncludePicture[horizontal][diagrams/signal-delivery.pdf]

\SlideTitle {Отправка сигналов через shell}
Процесс запускается в фоновом режиме с помощью символа {\tt &}.
\starttyping
$ sleep 100 &
[1] 7584
$ ps -A | grep sleep
 7584 pts/42   00:00:00 sleep
$ kill 7584
$
[1]+  Complété              sleep 100
\stoptyping

\SlideTitle {Отправка сигналов, API}
\starttyping
	pid_t pid;

	pid = fork();

	if (pid > 0) {
		int child_status;
		/* Родительский процесс. */

		/* Отправка сигнала SIGTERM дочернему процессу. */
		kill(pid, SIGTERM);

		/* Ожидание завершения дочернего процесса. */
		waitpid(pid, &child_status, 0);
	}
\stoptyping

\SlideTitle {Обработчики сигналов}

У всех сигналов есть обработчик по умолчанию:
\startitemize
\item SIGTERM, SIGKILL, SIGINT - процесс немедленно завершается,
\item SIGCHLD - сигнал игнорируется.
\stopitemize

Многие сигналы могут быть перехвачены процессом за счет регистрации своего
специального обработчика. Некоторые сигналы (например, SIGKILL) не
перехватываются.

Обработчик сигнала - функция на C. Но не все функции можно использовать в
обработчике сигналов.

\SlideTitle {Обработчики сигналов, API}
\starttyping
static void sigterm_handler(int sig)
{
	char *text = "Обрабатываю сигнал SIGTERM\n";
	write(1, text, strlen(text));
}

int main(int argc, char *argv[])
{
	struct sigaction sa;

	sa.sa_handler = sigterm_handler;

	/* int sigaction(int sig, const struct sigaction *act,
	                          struct sigaction *old_act); */
	sigaction(SIGTERM, &sa, NULL);
}
\stoptyping

\SlideTitle {Реентерабельные функции}
Не допускает повторное вхождение:
\starttyping
int t;

void swap(int *x, int *y)
{
	t = *x;
	*x = *y;
	*y = t;
}
\stoptyping

Допускает повторное вхождение:
\starttyping
void swap(int *x, int *y)
{
	int t;
	t = *x;
	*x = *y;
	*y = t;
}
\stoptyping

\SlideTitle {Сигналы, повторные вхождения}
\starttyping
static void sigterm_handler(int sig)
{
	/* Повторное вхождение в функцию printf нарушает внутреннее состояние
	потока вывода stdout. */
	printf("Обрабатываю сигнал SIGTERM\n");
}

int main(int argc, char *argv[])
{
	struct sigaction sa;

	sa.sa_handler = sigterm_handler;

	sigaction(SIGTERM, &sa, NULL);

	/* Функция printf() выполняет вывод данных через системные вызовы
	write(). Один из вызовов прерывается сигналом SIGTERM. */
	printf("Hello world\n");
}
\stoptyping

\SlideTitle {Отложенная обработка сигналов, код EINTR}
\starttyping
bool something_happened;

static void signal_handler(int sig)
{
	something_happened = true;
}

static int deferred_handler(void) {}

int main(int argc, char *argv[])
{
	/* ... Регистрация обработчика signal_handler(). */

	while (1) {
		char buf[100];
		int ret;

		ret = read(STDIN_FILENO, buf, sizeof(buf));
		if (ret < 0 && errno == EINTR) {
			deferred_handler();
		}
	}
}
\stoptyping

\SlideTitle {Ожидание завершения дочернего процесса}
Возможности.
\startitemize
\item Вызов {\tt waitpid()} без опции {\tt WNOHANG} - функция выполняется до
изменения состояния указанного процесса, блокируя решение других задач.
\item Обработка сигнала SIGCHLD - сигнал отправляется родительскому процессу в
случае завершения дочернего процесса.
\stopitemize

Как обрабатывать SIGCHLD.
\startitemize
\item Проверить статус всех дочерних процессов с помощью {\tt waitpid()}.
\item Опция {\tt WNOHANG} не используется.
\item Необходимо иметь в виду, что несколько отправленных сигналов могут
сливаться в один полученный сигнал, т.е. по одному сигналу SIGCHLD может
завершиться несколько дочерних процессов.
\stopitemize
\SlideTitle {Наследование программного окружения}
\IncludePicture[horizontal][diagrams/inheritance.pdf]

\SlideTitle {init}
\IncludePicture[horizontal][diagrams/init.pdf]

\SlideTitle {Демоны (daemons)}
\IncludePicture[horizontal][diagrams/daemons.pdf]

\stopcomponent
