
\startcomponent switch-slides
\environment slides-env

\setupcombinations[distance=1cm]

\setupTitle
  [ title={ПО сетевых устройств},
   author={Трещановский Павел Александрович, к.т.н.},
     date={\date},
  ]
\placeTitle

\SlideTitle {Обработка трафика}
Задачи:
\startitemize
\item программная реализация коммутатор Ethernet или маршрутизатора IP,
\item списки контроля доступа (ACL - access control list),
\item глубокий анализ пакетов (DPI - deep packet inspection),
\item реализация служебных протоколов (LLDP, STP, UDLD и пр.),
\item шифрование трафика
\item и др.
\stopitemize

\SlideTitle {Уровень сетевых интерфейсов}
\IncludePicture[horizontal][diagrams/low-level-traffic.pdf]

\SlideTitle {Сокеты семейства AF_PACKET}
\starttyping
	int sock;
	struct sockaddr_ll addr;

	/* AF_PACKET - семейство адресов канального уровня;
	 * SOCK_RAW - тип сокета, при котором приложению передается вместе с
	 * данными заголовок канального уровня;
	 * EHT_P_ALL - принимать данные любого протокола сетевого уровня. */
	sock = socket(AF_PACKET, SOCK_RAW, htons(ETH_P_ALL));

	memset(&addr, 0, sizeof(bind_addr));
	addr.sll_family   = AF_PACKET;
	addr.sll_ifindex  = ifindex; /* Индекс сетевого интерфейса */
	/* addr.sll_addr = ... Адрес канального уровня (здесь не используется).
	   addr.sll_halen = ...  */
	bind(sock, (struct sockaddr*)&addr, sizeof(addr));
\stoptyping

\SlideTitle {Структура Ethernet-кадра}
\externalfigure[diagrams/ethernet-frame.pdf][width=\textwidth]
\startitemize
\item Если значение Ethertype/Length меньше 0x800, то поле содержит длину
кадра. В противном случае - тип кадра (IP, ARP и т.д.).
\item CRC32 обычно устанавливается и проверяется на аппаратном уровне.
\item Размер кадра - от 64 до 1522 байтов.
\item ff:ff:ff:ff:ff:ff - широковещательный адрес. Кадр с таким адресом
назначения доставляется всем узлам сети.
\stopitemize

\SlideTitle {Стандартная структура {\tt struct ethhdr}}
\starttyping
#define ETH_ALEN 6

struct ethhdr {
	uint8_t h_dest[ETH_ALEN];
	uint8_t h_source[ETH_ALEN];
	uint16_t h_proto;
};

struct ether_addr {
	uint8_t ether_addr_octet[ETH_ALEN];
};
extern char *ether_ntoa(const struct ether_addr *addr);

void some_function(void) {
	struct ethhdr *eth_header;
	{
		char sa_str[20], da_str[20];
		printf("Source: %s destination %s\n",
		       ether_ntoa_r((struct ether_addr *)&eth_header->h_source, sa_str),
		       ether_ntoa_r((struct ether_addr *)&eth_header->h_dest, da_str));
	}
}
\stoptyping

\SlideTitle {Прием кадров из сокета AF_PACKET}
\starttyping
	int sock;
	struct sockaddr_ll rx_addr;
	socklen_t addrlen = sizeof(src_addr);
	char pkt_buf[2048];
	struct ethhdr *eth_header;
	struct ether_addr da;

	sock = socket(AF_PACKET, SOCK_RAW, 0);

	...

	/* rx_addr - информация об отправителе кадра. */
	recvfrom(sock, pkt_buf, sizeof(pkt_buf), 0, &rx_addr, &addrlen);

	/* Игнорируем передаваемые кадры. */
	if (rx_addr.sll_pkttype == PACKET_OUTGOING)
		return 0;

	eth_header = pkt_buf;

	memcpy(&da, eth_header->h_dest, ETH_ALEN);
	printf("Destination address: %s\n", ether_ntoa(&da));
\stoptyping

\SlideTitle {Коммутаторы Ethernet}
\startcombination
\framed{\externalfigure[diagrams/ethernet-switch.jpg][width=0.55\textwidth]}{}
\framed{\externalfigure[diagrams/rj-45.jpg][width=0.4\textwidth]}{}
\stopcombination

\SlideTitle {Канальный и физический уровни}
\IncludePicture[horizontal][diagrams/frame-transmission.pdf]

\SlideTitle {Терминология}
\startitemize
\item В сети Ethernet передаются {\em кадры} (англ. frame). В IP - дейтаграммы,
TCP - сегменты и др.
\item Узлы объединяются в сеть Ethernet с помощью {\em повторителей} и {\em
коммутаторов} (switch).
\item Коммутация производится между {\em портами}.
\item MAC (Media Access Control) - доступ к среде передачи. Название уровня
абстракции в стандарте Ethernet. Существуют различные методы доступа в как
сетях Ethernet, так и в сетях на основе других технологий.
\item MAC-адрес - адрес канального уровня. Обычно термин применяется к
Ethernet.
\item CSMA/CD (Carrier Sense Multiple Access with Collision Detection) - один
из методов доступа в сети Ethernet.
\stopitemize

\SlideTitle {Сеть на основе повторителя}
\IncludePicture[horizontal][diagrams/repeater.pdf]

\SlideTitle {Передача кадров через повторитель}
\IncludePicture[horizontal][diagrams/repeater-transmission.pdf]

\SlideTitle {Замечания}
\startitemize
\item Повторитель не буферизирует и не анализирует принимаемые принимаемые
кадры. Принимаемые данные побайтово повторяются на выходе.
\item Повторитель всегда передает данные на все выходные порты.
\item Одновременно передавать данные может только один узел сети. Если
несколько узлов начинают передачу в один момент времени, то наступает {\em
коллизия} и передача прекращается. Повторная попытка передачи производится
после случайной паузы.
\item Коллизии и широковещательные рассылки не позволяют полностью использовать
пропускную способность сети.
\item Повторитель не имеет своего MAC-адреса и не \quote{виден} другим
устройствам сети.
\stopitemize

\SlideTitle {Сеть на основе коммутатора}
\IncludePicture[horizontal][diagrams/switch.pdf]

\SlideTitle {Передача кадров через коммутатор}
\IncludePicture[horizontal][diagrams/switch-transmission.pdf]

\SlideTitle {Очереди в коммутаторе}
\IncludePicture[horizontal][diagrams/switch-queues.pdf]

\SlideTitle {Правила работы таблицы MAC-адресов}
\startitemize
\item Каждая запись в таблице - кортеж (MAC-адрес, номер порта).
\item При приеме каждого кадра производится изучение (learning) его адреса {\em
источника}. Для него создается запись в таблице, куда заносится номер входного
порта.
\item Если запись уже была, номер порта обновляется.
\item Далее по таблице производится определение выходного порта. Для
осуществляется поиск записи по адресу {\em назначения} принятого кадра. Если
запись есть, то номер порта в ней является номером выходного порта для кадра.
\item Если записи нет, то кадр передается на все порты, кроме входного.
\item Поиск выходного порта не производится для широковещательных кадров.
\stopitemize

\SlideTitle {Передача кадров при устаревании адреса}
\IncludePicture[horizontal][diagrams/switch-aging.pdf]

\SlideTitle {Замечания}
\startitemize
\item Таблица MAC-адресов позволяет уменьшить количество широковещательных
рассылок. Благодаря этому коммутатор эффективнее использует пропускную
способность сети.
\item Недостаток таблицы MAC-адресов - при изменении топологии сети часть
записей становится недействительными. Из-за этого возможна потеря связи между
узлами.
\item Смысл устаревание записей в таблице - гарантировать удаление
недействительных записей и последующее создание новых (действительных) записей.
\item Так как коммутатор анализирует заголовок кадра, все принимаемые кадры
буферизируются (принцип работы store and forward).
\stopitemize

\SlideTitle {Изменение топологии}
\IncludePicture[horizontal][diagrams/switch-topology-change.pdf]

\SlideTitle {Передача кадров при изменении топологии}
\IncludePicture[horizontal][diagrams/topology-change.pdf]

\SlideTitle {Программная реализация}
\startitemize
\item Коммутатором является приложение Linux.
\item Портами коммутатора являются сетевые интерфейсы. Прием и передача
осуществляется через сокеты AF_PACKET.
\item Входной очередью порта является очередь пакетов сокета.
\item Таблица MAC-адресов реализуется с помощью словаря {\tt struct avl_tree}.
Ключом является MAC-адрес.
\item Прием кадров и устаревание адресов осуществляется с помощью цикла
обработки событий.
\stopitemize

\SlideTitle {Виртуальная сеть}
\IncludePicture[horizontal][diagrams/virtual-topology.pdf]

\SlideTitle {Сетевые пространства имен}
\startitemize
\item Пространства имен - механизм виртуализации, разделяющий ресурсы
операционной системы на несколько частей. Доступ к каждой из них разрешается
только определенной группе процессов. Примеры ресурсов: номера пользователей,
файловая система, средства межпроцессного обмена.
\item Сетевое пространство имен виртуализирует сетевой стек Linux (таблицу
маршрутизации, сокеты и др.). Каждый сетевой интерфейс доступен только в одном
сетевом пространстве имен.
\item Каждый процесс находится в одном пространстве имен каждого типа.
\item Назначения пространства имен для каждого типа происходит независимо. 2
процесса могут находится в разных пространствах типа A, но при этом в одном
пространстве типа B.
\item Если у группы процессов совпадают все типы пространств имен, то эта
группа находится в {\em контейнере}.
\item В отличие от виртуальной машины, которая виртуализирует весь компьютер,
контейнер виртуализирует среду исполнения процесса.
\stopitemize

\SlideTitle {Виртуальные интерфейсы}
\IncludePicture[horizontal][diagrams/virtual-interfaces.pdf]

\SlideTitle {Запуск приложений в определенном пространстве имен}
Вывод списка интерфейсов в сетевом пространстве имен node_a:
\starttyping
# ip netns exec node_a ip addr
1: lo: <LOOPBACK> mtu 65536 qdisc noop state DOWN qlen 1000
    link/loopback 00:00:00:00:00:00 brd 00:00:00:00:00:00
4: eth0@if39: <NO-CARRIER,BROADCAST,MULTICAST,UP,M-DOWN> ...
    link/ether 68:eb:c5:00:01:02 brd ff:ff:ff:ff:ff:ff
    inet 192.168.20.6/24 scope global ge2
       valid_lft forever preferred_lft forever
\stoptyping
Запуск командного интерпретатора в пространстве имен node_a:
\starttyping
# ip netns exec node_a bash
\stoptyping
Замечание. Имена node_a и eth_a специфичны для AngtelOS. На других системах
именование пространств имен будет отличаться.

\SlideTitle {Реализация таблицы адресов с помощью словаря}

\starttyping
/* Функция сравнения узлов двоичного дерева. */
int avl_maccmp(const void *k1, const void *k2, void *ptr)
{
	const struct ether_addr *mac1 = k1, *mac2 = k2;
	int64_t mac1_int = 0, mac2_int = 0;

	memcpy(&mac1_int, mac1, sizeof(*mac1));
	memcpy(&mac2_int, mac2, sizeof(*mac2));
	if (mac1_int - mac2_int > 0)
		return 1;
	else if (mac1_int - mac2_int < 0)
		return -1;
	else
		return 0;
}

int main(int argc, char *argv[])
{
	struct avl_tree mac_table;
	/* avl_maccmp вместо avl_strcmp! */
	avl_init(&mac_table, avl_maccmp, false, NULL);
}
\stoptyping

\SlideTitle {Структура программного коммутатора}
\IncludePicture[horizontal][diagrams/software-bridge.pdf]

\SlideTitle {Алгоритм обработки кадра}
\IncludePicture[horizontal][diagrams/frame-handling-algorithm.pdf]

\SlideTitle {Протоколы ARP и ICMP}
\startitemize
\item Утилита ping использует протокол ICMP. ping отправляет запрос (echo
request) и ожидает получить ответ (echo reply). ICMP пакеты передаются внутри
IP-дейтаграмм.
\item ping отправляет запрос на определенный IP-адрес.
\item Для передачи IP-дейтаграммы ядро должно знать соответствие между
IP-адресом адресата и его MAC-адресом. Это соответствие устанавливает протокол
ARP.
\item Если в какой-то момент IP-адрес назначения неизвестен, ядро автоматически
отправляет ARP-запрос. Адресат отправляет ARP-ответ, содержащий его MAC-адрес.
\item Полученная информация заносится в ARP-таблицу в ядре.
\item Записи в ARP-таблице устаревают через определенное время.
\stopitemize

\SlideTitle {Замечания по задаче}
\startitemize
\item Порты коммутатора задаются через аргументы командной строки. Для каждого
порта указывается ifindex соответствующего интерфейса.
\item ifindex интерфейсов можно узнать командой {\tt ip addr}.
\item Работа коммутатора проверяется утилитами ping и tcpdump.
\item ping запускается в отдельном сетевом пространстве имен. Аргументом
является IP-адрес из другого пространства, {\em не} IP-адрес коммутатора. 
\item При реализации повторителя возможность коллизий игнорируется.
\stopitemize

\stopcomponent
