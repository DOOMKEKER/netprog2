
\startcomponent concurrency-slides
\environment slides-env

\setupTitle
  [ title={ПО сетевых устройств},
   author={Трещановский Павел Александрович, к.т.н.},
     date={\date},
  ]
\placeTitle

\SlideTitle {Сервер на основе блокирующих вызовов}
\starttyping
int handle_requests(int data_sock)
{
	while (1) {
		char rx_buf[100];

		ret = recv(data_sock, rx_buf, sizeof(rx_buf), 0);
		if (ret == 0)
			break;

		/* Обработка запроса. */
	}
}

int handle_connections(int listen_sock)
{
	while (1) {
		int data_sock;

		data_sock = accept(listen_sock, NULL, NULL);

		ret = handle_requests(data_sock);
	}
}
\stoptyping

\SlideTitle {Замечания}
\startitemize
\item {\tt recv()} и {\tt accept()} исполняются неопределенное время - до
появления новых данных/соединений.
\item Сервер не является многозадачным - не поддерживается одновременная
обработка из нескольких соединений.
\item Многие системные вызовы, отвечающие за ввод-вывод, являются блокирующими:
{\tt read()}, {\tt write()}, {\tt open()}, {\tt connect()}, {\tt send()}.
\item Как реализовать таймауты? Например, если мы хотим закрывать соединение, в
котором какое-то время не было запросов.
\stopitemize

\SlideTitle {Неблокирующий режим файлового дескриптора}
\starttyping
{
	int sock;
	int flags;

	sock = socket(AF_INET, SOCK_STREAM, 0);

	flags = fcntl(sock, F_GETFL);
	flags |= O_NONBLOCK;
	fcntl(sock, F_SETFL, flags);

	/* Теперь весь ввод-вывод через sock является неблокирующим. */
}

{
	int sock;

	sock = socket(AF_INET, SOCK_STREAM, 0);

	/* Отдельный неблокирующий вызов. */
	send(sock, tx_buf, sizeof(tx_buf), MSG_DONTWAIT);
}
\stoptyping

\SlideTitle {Использование неблокирующих вызовов}
\starttyping
int handle_requests(int data_sock)
{
	while (1) {
		char rx_buf[100];

		ret = recv(data_sock, rx_buf, sizeof(rx_buf), 0);
		if (ret == 0) {
			break;
		} else if (ret < 0 && errno == EWOULDBLOCK) {
			printf("No request\n");
		
			/* Что делать дальше? */
		} else if (ret < 0) {
			fprintf(stderr, "Failed to read request\n");
			return -1;
		}

		/* Обработка запроса. */
	}
}
\stoptyping

\SlideTitle {Возможное решение: многопоточность}
\startitemize
\item Каждое соединение обрабатывается отдельным потоком (\quote{нитью}).
\item В Linux поток является специальным процессом.
\item Все потоки имеют общее адресное пространство с основным процессом.
\item Для создания потока используется функция {\tt pthread_create()} вместо
{\tt fork()/execv()},
\item Стандартная библиотека предоставляет набор функций для работы с потоками:
{\tt pthread_join()}, {\tt pthread_cancel()}, {\tt pthread_attr_init()} и др.
\item Потоки исполняются под управлением планировщика процессов Linux.
\item Linux автоматически распределяет потоки pthread по имеющимся процессорным
ядрам.
\stopitemize

\SlideTitle {Многопоточный сервер}
\starttyping
int data_socks[MAX_CONNECTIONS];
int next;

void handle_requests(void *data)
{
	int data_sock = (int)data;

	while (1) {
		/* Чтение и обработка запроса. */
	}
}

int handle_connections(int listen_sock)
{
	while (1) {
		pthread_t thread;

		data_socks[next] = accept(listen_sock, NULL, NULL);
		pthread_create(&thread, NULL, handle_requests, (void *)data_socks[next]);
		next++;
	}
}
\stoptyping

\SlideTitle {Проблема: конфликт при доступе к общим данным}

\starttabulate[|lT|lT|lT|]
\NC Поток 1 \NC Поток 2 \NC Результат \NR
\NC array[0] = 'a'; \NC \NC {'a', '-'} \NR
\NC \NC array[0] = 'b'; \NC {'b', '-'} \NR
\NC \NC array[1] = 'b'; \NC {'b', 'b'} \NR
\NC array[1] = 'a'; \NC \NC {'b', 'a'}  - недопустимая комбинация\NR
\stoptabulate

\blank
Область кода, в которой происходит обращение к общим данным, называется {\em
критической секцией}. Чтобы обеспечить целостность данных, критическая секция
должна исполняться не более чем одним потоком в каждый момент времени.
\blank
Вход в критическую секцию защищается с помощью мьютексов:
\starttyping
mutex_lock();
array[0] = 'a'; array[1] = 'a';
mutex_unlock();
\stoptyping
Если при входе в критическую секцию поток обнаруживает запертый мьютекс,
он ждет, когда другой поток освободит этот мьютекс.

\SlideTitle {Многопоточный сервер с мьютексами}
\starttyping
pthread_mutex_t global_mutex;

void handle_requests(void *data)
{
	int data_sock = (int)data;
	while (1) {
		recv(data_sock, rx_buf, sizeof(rx_buf), 0);

		pthread_mutex_lock(&global_mutex);
		/* Обработка запроса. */
		pthread_mutex_unlock(&global_mutex);
	}
}

int handle_connections(int listen_sock)
{
	while (1) {
		pthread_t thread;
		data_socks[next] = accept(listen_sock, NULL, NULL);
		pthread_create(&thread, NULL, handle_requests, (void *)data_socks[next]);
	}
}
\stoptyping

\SlideTitle {Нужен ли вообще параллелизм?}
\hfil \externalfigure[diagrams/parallelism-redundancy.pdf][height=0.7\textheight] \hfil
\blank

parallelism - решение нескольких задач на нескольких ядрах.

concurrency - исполнение нескольких задач поочередно на одном ядре.

\SlideTitle {Использование гипотетического кооперативного планировщика}
\starttyping
void handle_requests(void *data)
{
	int data_sock = (int)data;
	while (1) {
		ret = recv(data_sock, rx_buf, sizeof(rx_buf), 0);
		if (ret < 0 && errno == EWOULDBLOCK)
			schedule();

		/* Обработка запроса. */
	}
}

int handle_connections(int listen_sock)
{
	while (1) {
		cooperative_thread_t thread;
		data_socks[next] = accept(listen_sock, NULL, NULL);
		if (data_socks[next] < 0 && errno == EWOULDBLOCK)
			schedule();
		coop_thread_create(&thread, NULL, handle_requests, (void *)data_socks[next]);
	}
}
\stoptyping

\SlideTitle {Вызов функций по событиям}
\starttyping
void event_dispatcher(void)
{
	int listen_sock, data_socks[MAX_CONNECTIONS];

	/* wait_events - гипотетическая функция, которая ждет появления новых
	соединений или данных хотя бы на одном из перечисленных сокетов. */
	wait_events(listen_sock, data_sock[0], ... , data_socks[MAX_CONNECTIONS - 1]);

	/* has_events - гипотетическая функция, которая проверяет наличие новых
	соединений или данных на указанном сокете. */
	if (has_events(listen_sock))
		handle_connections(listen_sock);

	for (i = 0; i < MAX_CONNECTIONS; i++)
		if (has_events(data_socks[i]))
			handle_requests(data_socks[i]);
}
\stoptyping

\SlideTitle {Ожидание событий с помощью {\tt select()}}
\starttyping
int select(int nfds, fd_set *readfds, fd_set *, fd_set *, struct timeval *timeout);
\stoptyping
\startitemize
\item {\tt fd_set} - {\em множество файловых дескрипторов}.
\item Функция {\tt select()} принимает на вход три множества дескрипторов и
ждет до тех пор, пока хотя бы один дескриптор в любом множестве не окажется в
состоянии {\em готовности}. Для множества readfds функция ожидает {\em
готовность к чтению}.
\item Функция select() меняет множества дескрипторов. После завершения в каждом
множестве остаются только дескрипторы в состоянии готовности.
\item Готовность к чтению означает, что последующий {\tt read} (или {\tt recv})
вернет данные без ожидания.
\item Аргумент {\tt timeout} задает максимальное время ожидания готовности.
Если это время истекает, {\tt select} возвращает 0.
\item Если {\tt timeout} установлен в {\tt NULL}, {\tt select} ждет готовности
неограниченное время.
\stopitemize

\SlideTitle {Работа с множеством файловых дескрипторов {\tt fd_set}}
\startitemize
\item {\tt fd_set} - битовая маска.
\item Задание множества:
\starttabulate[|lT|lT|]
\NC fd_set fds; \NC \NR
\NC int fd1 = 2, fd2 = 5; \NC \NR
\NC FD_ZERO(&fds); \NC $...00000000_2$ \NR
\NC FD_SET(fd1, &fds); \NC $...00000100_2$ \NR
\NC FD_SET(fd2, &fds); \NC $...00100100_2$ \NR
\stoptabulate
\item Проверка принадлежности множеству:
\starttyping
int fd1 = 2;
if (FD_ISSET(fd1, &fds)) {
	/* Если принадлежит. */
} else {
	/* Если не принадлежит. */
}
\stoptyping
\item Аргумент {\tt nfds} должен быть равен {\tt (max(fd1, fd2, ..., fdn) + 1)}.
\stopitemize

\SlideTitle {Вызов функций по событиям 2}
\starttyping
void event_dispatcher(void)
{
	int listen_sock, data_socks[MAX_CONNECTIONS], max_fd;
	fd_set fds;

	FD_ZERO(&fds);
	FD_SET(listen_sock, &fds);
	max_fd = listen_sock;
	for (i = 0; i < MAX_CONNECTIONS; i++) {
		FD_SET(data_socks[i], &fds);
		if (data_socks[i] > max_fd)
			max_fd = data_socks[i];
	}

	select(max_fd + 1, &fds, NULL, NULL, NULL);

	if (FD_ISSET(listen_sock, &fds))
		handle_connections(listen_sock);

	for (i = 0; i < MAX_CONNECTIONS; i++)
		if (FD_ISSET(data_socks[i], &fds))
			handle_requests(data_socks[i]);
}
\stoptyping

\SlideTitle {Работа со структурой {\tt struct timeval}}
\startitemize
\item Определение
\starttyping
struct timeval {
	time_t      tv_sec;     /* seconds */
	suseconds_t tv_usec;    /* microseconds */
};
\stoptyping
\item Получение текущего времени:
\starttyping
struct timeval tv;
gettimeofday(&tv, NULL); /* Время от начала Эпохи (1970 год) */
\stoptyping

\item Арифметика:
\starttyping
struct timeval tv1, tv2, newtv;
timeradd(&tv1, &tv2, &newtv); /* newtv = tv1 + tv2 */
timersub(&tv1, &tv2, &newtv); /* newtv = tv1 - tv2 */
if (timercmp(&tv1, &tv2, <) {
	/* tv1 < tv2 */
} else {
	/* tv1 >= tv2 */
}
\stoptyping
\stopitemize

\SlideTitle {Ожидание запроса с таймаутом}
\starttyping
/* Время, начиная с которого мы считаем соединение data_socks[0] устаревшим. */
struct timeval connection_expiration_time;

{
	struct timeval now, timeout;

	FD_ZERO(&fds);
	FD_SET(data_socks[0], &fds);
	max_fd = data_socks[0];

	gettimeofday(&now, NULL);
	/* timeout = connection_expiration_time - now */
	timersub(&connection_expiration_time, &now, &timeout);

	ret = select(max_fd + 1, &fds, NULL, NULL, &timeout);
	if (ret == 0) {
		/* Таймаут. */
		close(data_socks[0]);
		return 0;
	}
	if (FD_ISSET(data_socks[0], &fds))
		handle_requests(data_socks[0]);
}
\stoptyping

\SlideTitle {Расчет таймаута при наличии нескольких соединений}
\externalfigure[diagrams/timeout-calculation.pdf][width=\textwidth]
$timeout = min(Exp_1 - now, Exp_2 - now, ..., Exp_n - now)$

\SlideTitle {Работа с несколькими таймерами}
\starttyping
/* Время, начиная с которого мы считаем соединение data_socks[i] устаревшим. */
struct timeval connection_expiration_times[MAX_CONNECTIONS]

{
	struct timeval now, timeout = {.tv_sec = LONG_MAX}, candidate;

	gettimeofday(&now, NULL);
	for (i = 0; i < MAX_CONNECTIONS; i++) {
		timersub(&connection_expiration_times[i], &now, &candidate);
		if (timercmp(&candidate, &timeout, <)
			timeout = candidate;
	}

	ret = select(max_fd + 1, &fds, NULL, NULL, &timeout);
	if (ret == 0) {
		/* Могло пройти больше времени, чем timeout. */
		gettimeofday(&now, NULL);
		for (i = 0; i < MAX_CONNECTIONS; i++) {
			if (timercmp(&connection_expiration_times[i], &now, <)
				close(data_socks[i]);
	}
}
\stoptyping

\SlideTitle {Цикл обработки событий (event loop)}
\starttyping
while (1) {
	struct fd_set fds;
	struct timeval timeout;

	FD_ZERO(&fds);
	for (...) { /* Перебор всех сокетов */
		FD_SET(sock_fd, &fds);
		max_fd = sock_fd > max_fd ? sock_fd : max_fd;

		... /* Модификация timeout */
	}
	ret = select(max_fd + 1, &fds, NULL, NULL, &timeout);
	if (ret == 0) {
		/* Обработка таймаута. */
	} else {
		for (...) { /* Перебор всех сокетов */
			if (FD_ISSET(sock_fd, &fds)) {
				/* Обработка нового соединения или запроса. */
			}
		}
	}
}
\stoptyping

\SlideTitle {Правила написания обработчиков событий}

\startitemize
\item Не должно быть блокирующих вызовов.
\item Не должно быть больше одного неблокирующего вызова, который может вернуть
ошибку {\tt EWOULDBLOCK}.
\stopitemize

Так делать нельзя:
\starttyping
int handle_requests(int data_sock)
{
	recv(data_sock, request1, sizeof(request1), MSG_DONTWAIT);
	/* Обработка запроса 1. */
	send(data_sock, response1, sizeof(response1), 0);

	recv(data_sock, request2, sizeof(request2), MSG_DONTWAIT);
	/* Обработка запроса 2. */
	send(data_sock, response2, sizeof(response2), 0);

	return 0;
}
\stoptyping

\SlideTitle {Правила написания обработчиков событий (2)}

\starttyping
int num_requests;

int handle_requests(int data_sock)
{
	switch (num_requests) {
	case 0:
		recv(data_sock, request1, sizeof(request1), MSG_DONTWAIT);
		/* Обработка запроса 1. */

		/* Предполагаем, что send() никогда не блокирует программу. */
		send(data_sock, response1, sizeof(response1), 0);
		nrequests++;
		break;
	case 1:
		recv(data_sock, request2, sizeof(request2), MSG_DONTWAIT);
		/* Обработка запроса 2. */
		send(data_sock, response2, sizeof(response2), 0);
		nrequests++;
		break;
	}
	return 0;
}
\stoptyping

\SlideTitle {Больше, чем просто {\tt select()}}
\startitemize
\item Помимо {\tt select()}, Linux предлагает функционально эквивалентные API
{\tt poll()} и {\tt epoll()}.
\item Функция {\tt select()} реализует шаблон (паттерн) проектирования
\quote{реактор}.
\item Библиотеки libevent и libev для разработки высокопроизводительных
серверов основаны на вызовах {\tt select()}/{\tt poll()}.
\item В большинстве языков программирования есть библиотеки, предоставляющие
аналогичные функции: Twisted в Python, EventMachine в Ruby и др.
\item В некоторых языках программирования (JavaScript) весь ввод-вывод
осуществляется через планирование и обработку событий.
\stopitemize

\SlideTitle {Диаграмма состояний}
\IncludePicture[horizontal][diagrams/state-diagram.pdf]

\SlideTitle {Конечный автомат}
Конечный автомат Мура: $(\Sigma, S, s_0, \delta, F)$
\startitemize
\item $\Sigma$ - множество входных событий.
\item $S$ - множество состояний.
\item $s_0$ - начальное состояние.
\item $\delta$ - функция переходов, $\delta: S \times \Sigma \rightarrow S$.
\item $F$ - множество конечных состояний.
\stopitemize

Автомат Мили: для каждого перехода между состояниями дополнительно задается
выходной сигнал. На диаграмме состояний событие и выходной сигнал разделяются
символом '/'.

\SlideTitle {Конечный автомат TCP соединения}
\IncludePicture[horizontal][diagrams/tcp-state-machine.pdf]

\SlideTitle {Пример: лампочка}
\IncludePicture[horizontal][diagrams/light-bulb.pdf]

\SlideTitle {Реализация (1)}
\starttyping
enum state {
	LIGHT_ON,
	LIGHT_OFF,
	BROKEN,
};

enum events {
	TURN_ON,
	TURN_OFF,
	BRICK,
};

static enum state state;
\stoptyping

\SlideTitle {Реализация (2)}
\starttyping
void run_state_machine(enum event event)
{
	switch (state) {
	case LIGHT_OFF:
		if (event == TURN_ON)
			state = LIGHT_ON;
		else if (event == BRICK)
			state = BROKEN;
		break;
	case LIGHT_ON:
		if (event == TURN_OFF)
			state = LIGHT_OFF;
		else if (event == BRICK)
			state = BROKEN;
		break;
	case BROKEN:
		break;
	}
}
\stoptyping

\SlideTitle {Реализация (3)}
\starttyping
int main(int argc, char *argv[])
{
	enum event event;
	
	while (1) {
		char command[128];

		fgets(command, sizeof(command), stdin);
		if (!strncmp(command, "turn-on", 7))
			event = TURN_ON;
		else if (!strncmp(command, "turn-off", 8))
			event = TURN_OFF;
		else if (!strncmp(command, "brick", 5))
			event = BRICK;

		run_state_machine(event);
	}

	return 0;
}
\stoptyping

\SlideTitle {Конечный автомат + select()}
\startitemize
\item  Поскольку переход между состояниями меняет внутреннее состояние
программы и обычно не требует блокирующих вызовов, функция перехода может
безопасно вызываться из цикла обработки событий. Это позволяет поочередно
исполнять несколько экземпляров одного конечного автомата в одном многозадачном
приложении. Каждый экземпляр в нем можно рассматривать как одну задачу.
\item Программа должна выполнять трансляцию низкоуровневых событий,
детектируемых с помощью функции select(), в высокоуровневые события, подаваемые
на вход конечного автомата. Примеры низкоуровневых событий: чтение запроса из
сокета, истечение таймаута, разрыв соединения, установка нового соединения.
Высокоуровневые события представляются переменными перечислимого типа.
\stopitemize

\SlideTitle {Многозадачный сервер}
\startitemize
\item Все входные сигналы (новые соединения, новые запросы, таймауты)
обрабатываются циклом обработки событий на основе вызова {\tt select()}.
\item Логика обслуживания каждого отдельного соединения описывается конечным
автоматом.
\item При установке нового соединения программа создает новый экземпляр
конечного автомата.
\item При получении нового запроса или при срабатывании таймаута программа
определяет экземпляр конечного автомата (соединение), к которому они относятся,
и передает их на вход автомата в качестве событий.
\item При переходе между состояниями автомата программа меняет внутреннее
состояние (задает переменные, устанавливает таймеры), а также формирует
выходные сигналы (ответы на запросы и т.д.).
\stopitemize

\SlideTitle {Задача: повторитель символов}
\IncludePicture[horizontal][diagrams/repeater.pdf]

\stopcomponent
